\section{Introduction}

Cloud computing has marked significant developments and possibilities in the industry. It focuses on offering services for the different needs of the modern society.


There are three categories of cloud computing services namely, Software as a Service (SaaS), Platform as a Service (PaaS) and Infrastructure as a Service (IaaS). Organizations provide Saas depending on the demand. Google Apps is one example of Saas that can be used to manage email and create documents, etc. Paas offers developers a platform where they can build and deploy applications. Iaas provides storage, servers and handling computer clusters. These tools are primarily created to serve computational needs \cite {Ahuja2012}. A cloud computing platform dynamically allocates, configures, reconfigures and deallocates servers as requested or on demand. This approach ensures the elasticity of cloud computing \cite {Brandic2011}.


Most scientific applications require high performance computing (HPC) which needs CPU intensive computations and large data storage. To be able to host such applications, several computers interconnected in a network such as clusters are needed. This makes scientific computing very costly. Fortunately, with the advancement in cloud computing, these tools can be deployed in the cloud without worrying about the costs of hardware purchase and maintenance \cite {Ahuja2012}. To determine the cost of hosting scientific computing in cloud, the Scientific Computing Cloud (SciCloud) project is conducted in the University of Tartu. Results have shown that transmission delays are the major concern in pursuing HPC applications in the cloud\cite {Brandic2011}.  To address this problem, a study is conducted by Hermocilla\cite {Hermocilla2014} in Peak-Two Cloud (P2C). One of the features introduced by P2C is vCluster. vCluster is a tool that enables a user to deploy a working (MPI) cluster on demand and to terminate it after execution. It uses a master-slave design implementation. However, vCluster is a command-line based application which has limited capacities for data manipulation, analysis, and presentation. Creating a web application that would host vCluster with additional features like graphical representation to visualize trends, and a Graphical User Interface(GUI) to offer convenient access. This paper will present SkyLab, a web application that aims to serve and to improve vCluster functionalities.

This study will help us visualize how HPC applications can be hosted in the cloud and understand why is this technology timely relevant.  Even with capable hardware, computations might take hours or even days which makes it a hassle for users. Thus, there is a need for a more convenient way of access for HPC applications. Furthermore, this would encourage students to explore and research on HPC applications. HPC applications usually have a plain numerical output. A means of presenting data graphically would make the interpretation of data more efficient.

The study would contribute to research on hosting HPC applications in the cloud. Problems encountered and solutions offered throughout the study will give insights to future researchers. Furthermore, the output application of the study can be used by the academe.

The study focuses on vCluster and is not mainly concerned on P2C as a whole. The study initially bases on the current implementation of vCluster and eventually improve it and offer additional features\cite {Hermocilla2014}.

This study aims to develop SkyLab, a web application that would function as a front-end for vCluster. 	Specifically, it should be able to: 
	\begin{enumerate}
		\item allow users to select tools and execute them via web interface; 
        \item create an extensible platform that would accommodate additional tools; and
		\item display output data of tools.
	\end{enumerate}

\subsubsection*{Functional Requirements}
	\begin{itemize}
		\item Users can upload input data to the server.
    	\item Users can view the results of the used tools.
    	\item The system is integrated with vCluster functionalities.
    \end{itemize}

\subsubsection*{Non-functional Requirements}
    \begin{itemize}
    	\item Users must be authenticated to be able to use the system.
       	\item The system must be easy to use and user friendly.
       	\item The system must be highly maintainable for future development.  
	\end{itemize}
        
\section {Related Work}
  
\subsection{Performance of HPC Applications in the Cloud}

Shajulin \cite {8521133920130201} has enumerated five performance concerns for HPC applications in the cloud. First concern is resource provisioning issues which includes memory management, energy management, response time, and resource outages.  Hardware virtualization for multiple guest operating systems running concurrently on a host computer creates a concern for huge overheads, delays, or memory leakages caused by the process of allocating memory, remapping it to virtual machines, and cleaning it up. The scenario of multiple guest OS can also cause an OS to fail to make hardware use energy-efficient because the presence of multiple hardware virtualizations prevent the main OS from having full control over actual hardware. Configuration failures, update failures, or even unknown issues cause resource outages. Second, most programming models used in the cloud have inconsistencies on error recovery and storage handling by design and by execution. If these inconsistencies are not addressed, the system components might fail because they are not well-integrated.        

Third, enforcing terms of Service Level Agreement (SLA) is bound to performance issues due to loss of control over resource provisioning or deploying cloud services. Fourth, there is a lack of standard conventions in clouds. This means that there is no guarantee that the current configuration of the an application will be compatible with other providers. The user might need to modify the program for a specific provider.  
         
Fifth, security measures challenge performance since increasing security protocols decreases response time.

In a study by Jackson et al.\cite{CloudCom2010} conducted on Amazon EC2, benchmarks are used to imitate the workload of a typical HPC application.

Results of the study show that the percentage of the communication time of the application is strongly correlated to its overall performance. Communication time and frequency of global communications are inversely proportional to the application's performance.

The more communication, the worse it performs. In addition, applications with significant global communication perform relatively worse than those with less. 

Lastly, performance variability is introduced by the added layers of abstraction caused by hardware virtualization.
         
Another study conducted by Juve et al.\cite{juve_scientific_2009}, analyzed running scientific workflows in Amazon EC2. It has been found that getting resources is the primary cost factor to consider in executing workflow tasks while storage has relatively cheaper costs. Storing data in the cloud rather that transferring it to each workflow effectively reduces the high cost of data transfer. These results show that the cloud is a good alternative for running scientific workflow applications but for cloud providers to be able compete with existing physical clusters in terms of performance of HPC applications would need high-speed networks and parallel file systems\cite{WalkerEC2HPC}. While this may mean that cloud computing is yet to support large-scale HPC applications, it can still be considered as an alternative for deploying simple HPC tasks on demand\cite{PerfAnalysisManyTasks}. Despite the performance trade-off, EC2 offers ease-of-use and cheaper costs which are marginal factors in consideration for academic purposes\cite{ZachHumphrey}.  
         
         
\subsection {Related Systems}

Ganglia is a system designed to monitor high performance computing systems. It uses a hierarchical model in managing the system of clusters. It uses optimized data structures and communication algorithms to achieve scalability with high concurrency. It is claimed to be used by over 500 clusters around the world. This implies that the system is tested and trusted to be used for real-world applications\cite{1395654820040701}.
	    
One of the main inspirations for developing SkyLab is the Yabi system. It provides a web interface with support for workflow environments with focus on introducing HPC applications to non-technical audience. Users can create and reuse workflows, and manage large amounts of data while system administrators can configure tools via the web interface as well. It is currently in use by multiple institutions, and is maintained as an open-source project\cite{7411021620120101}.	    	    
	    
Another related project is Web Interface for mpiBLAST (WImpiBLAST). It supports mpiBLAST, a parallel implementation of Basic Local Alignment Search Tool (BLAST). BLAST is a software used for sequence homology similarity search in large databases of gene sequences. mpiBLAST can utilize HPC clusters to achieve faster computing speeds but it requires knowledge in using MPI commands to benefit from its advantages. WImpiBLAST addresses this problem by providing the user a web interface to simplify the steps to use mpiBLAST\cite{9686120720140601}.   
            
        
\section{Design and Implementation}

\subsection{System Architecture}
SkyLab functions as a web front-end for vCluster which is built on top of Peak-Two Cloud.
	\begin{center}			
		\includegraphics[width=92px,height=224px]{./images/system_architecture.png}			
		\captionof{figure}{System Architecture of SkyLab}			
	\end{center}	
    

    %\includegraphics {./images/system_architecture.png}


%	\begin{center}			
%		\includegraphics[width=92px,height=224px]{./images/system_architecture.png}			
%		\captionof{figure}{System Architecture of SkyLab}			
%	\end{center}	
	
    \begin{figure*}[ht]
      \centering
      \includegraphics[width=500px,height=250px]{./images/use_case_large.png}
      \caption{Use Case Diagram of SkyLab}\label{System Architecture}
    \end{figure*}

\subsection{MPI clusters} 

The system spawns a thread (MPIThread) for each active cluster which handles the connection to the assigned cluster via Secure Shell (SSH). Creation and deletion of clusters is done by using vCluster commands while tool activation is done by using p2c-tools. The thread also manages task queuing and execution.
		
A cluster is either classified as public or private. If it is set to public, every user in the system can use it. On the other hand, for private clusters, the cluster will only be visible to the owner. The owner has the option to share the cluster to other users via the share key generated for the said cluster. 		

\subsection{Tool sets} 
The system searches for Python packages inside the assigned modules folder and install it on server start. The tool sets will then be available for use with the system. The package must have a Python module named \emph{install.py} which contains function calls for integrating the package with the system. The package must also contain the corresponding views and executable classes for each sub-tool.  

\subsection{Tasks} 
The system creates a task object for each task input by the user. A signal will then be sent and it is then received by the corresponding MPIThread which queues the task for execution. When a task is executed, it calls the assigned executable class with the given parameters. On connection error, the task waits exponentially before retrying. If the server crashes while running task execution, the task is just restarted.				
		
Default task execution flow via executable class:			
	\begin{enumerate}
		\item  Needed remote and local directories for execution are cleared or created.
		\item  Input files are uploaded to cluster.
		\item  List of commands given are executed.
		\item  Output files are sent back to the server.
		\item  Remote task folder is deleted.
		\item  Output files are served by the server.
	\end{enumerate}	

	
\subsection{Technologies}
	\begin{itemize}
   		\item Programming Language: Python 
   		\item Web Framework: Django  
   		\item DBMS: MariaDB  
   		\item Cloud Infrastructure: Peak-Two Cloud 
	\end{itemize}	
	
\subsection{Evaluation of Methodology}

The use cases given ensure that the user will be able to select tools and execute them with the system. Also, the output files of user tasks will be served by the system. The system code is required maintainable for future inclusion of other tools.  

The methodology provided is sufficient to achieve the given objectives of the study. The created software will be subjected to user acceptance test for further evaluation. 

\section{Results and Discussion}

\subsection{Supported Tools}
		
	\begin{description}
    	\item[AutoDock] \hfill \break 
        	It is a software used to simulate protein-ligand docking\cite{morris2009autodock4}.
        \item[AutoDock Vina] \hfill \break
            It is a software similar to AutoDock 4 but on the average, it provides faster and more accurate computations\cite{JCC:JCC21334}. 
            
            
            %It achieves significant improvements in the %average accuracy of the binding mode predictions, %while also being up to two orders of magnitude %faster than AutoDock 4
            
		\item[DOCK] \hfill \break
            It is used to predict the small molecule-target interactions\cite{lang2009dock}.
            
      	\item[Quantum ESPRESSO] \hfill \break
            It is an integrated software suite of tools for ab-initio molecular dynamics (MD) simulations and electronic structure calculations\cite{QE-2009}.

  		\item[GAMESS] \hfill \break
            It is used for ab initio molecular quantum chemistry. \cite{1993gamess}
            
 	    \item[Ray] \hfill \break
            It is uses parallel genome assemblies for parallel DNA sequencing \cite{boisvert_ray_2012}.
          \end{description}
          
\subsection{System Features}
The system's interface offers different functionalities that simplifies MPI cluster and task management.
		
	\begin{itemize}
		\item The user is authenticated by logging in with his @up.edu.ph Google account.
		\item The user can create an MPI cluster with optional tool activations. 
			\begin{center}			
				\includegraphics[scale=0.50]{./images/create_mpi.png}
				\captionof{figure}{MPI creation form}			
			\end{center}	
	
		\item The user can monitor visible public and private MPI clusters.  \newline	
		\begin{center}			
			\includegraphics[scale=0.33]{./images/mpi_list_view.png}
			\captionof{figure}{MPI cluster table}		
		\end{center}
		
	
		\item The user can make a private cluster visible by entering a valid share key. \newline
		\begin{center}			
			\includegraphics[scale=0.50]{./images/add_private_cluster_2.png}		
			\captionof{figure}{Add private cluster form}			
		\end{center}	
		
		\item The user can view details about a MPI cluster. If the user is the cluster's owner he has the option to delete the cluster. \newline
		\begin{center}			
			\includegraphics[scale=0.40]{./images/mpi_detail_view_2.png}			
			\captionof{figure}{MPI detail view}			
		\end{center}	
		
		\item The user can select from a list which tool does he want to use. 	
		\begin{center}			
			\includegraphics[scale=0.45]{./images/toolset_list_2.png}			
			\captionof{figure}{Tool set list view}			
		\end{center}
		
		\item The user can submit a task by filling up a tool's task creation form. \newline
		\begin{center}			
			\includegraphics[scale=0.40]{./images/gamess_form_2.png}			
			\captionof{figure}{GAMESS task creation form}			
		\end{center}	
		
		\item The user can monitor created tasks. \newline
		\begin{center}			
			\includegraphics[scale=0.40]{./images/tasks_list_view_2.png}			
			\captionof{figure}{Task table view}			
		\end{center}	
		\item The user can view results of tasks. JSmol renders the compatible output files\cite{IJCH:IJCH201300024}. \newline
		\begin{center}			
			\includegraphics[scale=0.35]{./images/jsmol_detail_view_2.png}			
			\captionof{figure}{Task detail view}			
		\end{center}	

		    
		
	\end{itemize}
	
	\subsection{User Acceptance}
	\begin{center}			
			\includegraphics[scale=0.32]{./images/uat_graph.png}			
			\captionof{figure}{Results of QUIS for SkyLab}			
	\end{center}
	The system has been evaluated by 56 respondents by answering a survey based on Questionnaire for User Interface Satisfaction (QUIS) \cite{chin1988development}. Respondents are students who are unfamiliar with both HPC tools and the concept of MPI systems.  Respondents are asked to test features of SkyLab by following a set of instructions and using input files provided. On the average, the users rated their overall experience with SkyLab to 6.9/10. The users listed the simplicity of the user interface to be the most positive aspect of the system while the slow speed of task processing is said to be the most negative. Majority of the tools supported by SkyLab have inherently long processing time which is not known to the respondents. The system does not focus on optimizing the said tools to achieve better performance but rather it focuses on simplifying the user's task submission process. 

	
\section{Conclusion and Future Work}
The system created allows users to manage MPI clusters and submit tasks without the need for technical expertise in scripting. This makes the advantages of HPC available to non-technical users. This is achieved by parsing form inputs to generate commands for task execution. Task files can be download from the server and output files are displayed with the help of JSmol\cite{IJCH:IJCH201300024}. The system is also configured to install tool sets found in the modules folder making it possible to accommodate additional tools. Based on the user acceptance test conducted, the users found the system to be acceptable in terms of the criteria provided, in general. 

The system achieved its main objectives but its features can still be improved and additional features can be introduced. Improved input parameter checking and error handling will make the system more robust. There are still use cases of tools that are yet to be supported. Input file generation can make the process more interactive and more customizable.  Workflow design support will enable users to run complex tasks. Support for custom MPI programs will make it easier for developers to utilize the system as a test environment. Task scheduling and resource management algorithms can be used to efficiently handle resource-intensive or time consuming tasks. For example, a cluster can borrow resources from idle clusters. These recommendations will provide the users a better experience in using the system for academic and research purposes. 

% APPENDICES
%\appendices

%\section{Proof of the First Zonklar Equation}
%Appendix one text goes here...

%\section{}
%Appendix two (without title) text goes here...

% ACKNOWLEDGMENT
\section*{Acknowledgment}

%Many thanks to...
% BIBLIOGRAPHY
% \nocite{*}

% BIOGRAPHY
%\begin{biography}[{\includegraphics{./yourPicture.eps}}]{Student M. Name}
%Biography text here...
%\end{biography}



